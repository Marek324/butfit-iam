\documentclass[fontsize=12pt]{article}
\usepackage{amsmath}
\usepackage{hyperref} 
\usepackage{fontsize}
\usepackage[margin=2cm]{geometry}

\title{IAM druhý domáci úkol - Krteček}
\author{Marek Hric xhricma00}
\date{27 April 2025}

\begin{document}

\maketitle
\fontsize{12pt}{14.4}

\section{Problém}
Při dělení jistého přirozeného čísla čísly 20 a 25 vyjdou jako zbytky prvočísla. Součet obou neúplných podílů se rovná 2025. Určete dělené číslo. Najděte všechna řešení.

\section{Pomocné zápisy}
\textbf{Prvočísla:}\\
$\-$\quad$\mathcal{P}=\{\forall p\in\mathcal{N}: (p>1)\wedge \forall d\in\mathcal{N}: (d\vert p\Rightarrow(d=1 \lor d=p)) \}$
\\
\textbf{Delenie so zvyškom:}\\
$\-$\quad$x\div y=z$ mod $r\Leftrightarrow x=y\cdot z+r$

\section{Prepis problému}
$x=20\cdot a+p_1=25\cdot b+p_2\wedge a+b=2025$ $;$ $x,a,b\in \mathcal{N}\wedge p_1,p_2\in \mathcal{P}$\\
Všeobecne:\\
$x=d_1\cdot a+p_1=d_2\cdot b+p_2\wedge a+b=n$ $;$ $x,a,b,d_1,d_2,n\in \mathcal{N}\wedge p_1,p_2\in \mathcal{P}$

\section{Riešenie}
\subsection{Obmedzenia}
Zo zadania vyplývajú tieto obmedzenia: 
\begin{itemize}
    \item $n,a,b,d_1,d_2\in\mathcal{N}$
    \item $p_1,p_2\in\mathcal{P}$
    \item $p_1\in\{\forall p\in\mathcal{P}: p < d_1\}$
    \item $p_2\in\{\forall p\in\mathcal{P}: p < d_2\}$
\end{itemize}
$d_1\cdot a + p_1 = d_2\cdot b + p_2 \Leftrightarrow
d_1\cdot a - d_2\cdot b = p_2 - p_1$\\
$d = GCD(d_1, d_2)$\\
$dd_1 = d_1\div d$\\
$dd_2 = d_2\div d$\\
$d\div (dd_1\cdot a - dd_2\cdot b) = p_2 - p_1 \Leftrightarrow
dd_1\cdot a - dd_2\cdot b = \dfrac{p_2 - p_1}{d} \Rightarrow$ 
\underline{$GCD(d_1, d_2)\vert (p_2 - p_1)$}

\subsection{Výpočet}
Výpočet je riešený python scriptom, ktorý berie do úvahy tieto obmedzenia a hľadá riešenia.
Tento script je dostupný na:
\\\href{https://github.com/Marek324/butfit-iam/blob/master/prime-remainders.py}{https://github.com/Marek324/butfit-iam/blob/master/prime-remainders.py}.

\subsection{Výsledok}
Zadaním zadané hodnoty:\\
$n = 2025$\\
$d_1 = 20$\\
$d_2 = 25$\\
Výsledok:\\
$\{22502, 22503, 22505, 22507, 22511, 22513, 22517, 22519\}$\\
\\\\Jednoduchým pozorovaním si môžeme všimnúť že všetky výsledky sú 22500 + prvočísla $<2-19>$, preto som sa skúšal zamyslieť nad nejakým vzorcom, ktorým by sme tento príklad mohli riešiť, ktorý som síce pre tento vstup dokázal nájsť, nanešťastie ale všeobecne nefunguje.  
\end{document}
